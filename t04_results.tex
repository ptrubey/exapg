\subsection{SLOSH}
\begin{figure}[ht]
    \caption{Locations of identified sites that experienced inundation in more than 
        $5$ percent of storms.  Specifically identified are two locations of 
        interest. \makenote{Change Label to PIA, add packer terminal, make plot 
        in-line to limit lost page space?} \label{map:delawarebay}}
    \centering
    \includegraphics[height=3in]{./plots/delaware}
\end{figure}

\makenote{rewrite paragraph}
The range of latitudes at which storms make landfall in the simulation cover a region of
    coastline surrounding the entrance of Delaware Bay \makenote{rephrase more accurate given
    data}.  Given that focus inherent in the original data, we focus our primary analysis there.
    Intersecting the original \emph{SLOSH} grid for the Delaware bay and surrounding watershed, 
    for grid cells that experienced inundation in more than 5 percent of simulations, 
    with specified locations, we identify 65 locations of interest, at which we will conduct 
    our analysis.  Figure~\ref{map:delawarebay} gives the locations of sites we identified, along
    with a classification of those sites.  Services includes emergency services, police, fire,
    and medical services.  Land includes village centers, major intersections, and named
    land features.  Transportation includes ferry landings, airports, heliports, and similar.
    Within that set, we further identify two locations of particular interest, 
    for which significant inundation can lead to catastrophic consequences:  
    Dover Air Force Base (Dover AFB), and the Philadelphia International 
    Airport (PIA). Dover AFB is situated next to Delaware Bay, with a direct approach from the
    open ocean.  PIA, on the other hand, is further upstream, adjacent to the Delaware River, 
    in the city of Philadelphia.  To reach this location, storm surge would need to backflow 
    the Delaware River a significant distance.

One difficulty with a high-dimensional model is evaluating its fidelity.  As we saw in the simulation
    study, the relative rise in energy score between a bad model and a good model shrinks as
    dimensionality increases.  This is related to the curse of dimensionality in applications like
    $k$-nearest neighbor algorithms: as the number of dimensions increases, the ratio in average
    distance between the nearest replicate, and farthest replicate in a sample will tend to approach
    1.  In this regard, distance, and metrics based on distance are fundamentally flawed in a 
    high-dimensional setting.\makenote{[Citation needed!]}

We can subjectively assess recovery of marginal empirical CDF, using marginal
    posterior predictive CDF's under our various modelling approaches.
    Having sampled $\bm{V}^{*}$ from its posterior predictive density, we can get a sample of $W_s^*$
    by inverting Equation~\eqref{eqn:standardization}.  Thus, for for $R^*\sim\text{Pareto}(1)$, 
    $Z^* = R^*\bm{V}^*$,
    \begin{equation*}
        W_s^* = a_s\left(\frac{(Z_s^*)^{\xi_s} - 1}{\xi_s}\right) + b_s
    \end{equation*}
    where $\bm{\xi}$, $\bm{a}$, and $\bm{b}$ were previously calculated.  
    For consistency with regard to the originating data, we truncate replicates 
    from the posterior predictive distribution $W_{s}^* \geq 0$.  
    \makenote{rest of paragraph potentially unnecessary}
    It can occur that the chosen marginal GP parameters are sub-optimal, as the threshold
    $b_{ts}$ may be ill-chosen.  This is an inherent weakness of choosing
    the threshold via the empirical quantile function, which creates a 
    tradeoff---to use some other method than the empirical quantile function would necessarily imply 
    non-uniform marginal probabilities of threshold exceedence, upon the assumption of which some other 
    aspects of our analysis are based. \makenote{[are they?]}

\begin{figure}[ht]
    \centering
    \begin{subfigure}{0.8\textwidth}
        \caption{Dover Air Force Base\label{plot:marginal_doverafb}}
        \includegraphics[width=\textwidth]{./plots/delaware_marginal_dover_afb.png}
    \end{subfigure}
    \\\vspace{1cm}~\\
    \begin{subfigure}{0.8\textwidth}
        \caption{Philadelphia International Airport\label{plot:marginal_pia}}
        \includegraphics[width=\textwidth]{./plots/delaware_marginal_phil_ia.png}    
    \end{subfigure}
    \caption{Empirical, and posterior predictive cumulative distribution functions for marginal 
    $v$, $V^*$ (Left) and $w$, $W^*$ (Right), at denoted locations, under various modeling
    considerations.\makenote{remove legend name, compress into single faceted plot per location. 
    make plot for packer terminal?}}
\end{figure}

In Figure~\ref{plot:marginal_doverafb}, we observe said marginal empirical and posterior-predictive
    CDF's for storm surge at Dover Air Force Base.  As previously mentioned, this location is adjacent
    to Delaware Bay, approximately 2 miles inland, with a direct line of sight out the mouth of the
    bay towards the open ocean.  Take note in particular, that the empirical CDF of $\bm{w}_s$ shows
    that storm surge does not reach Dover AFB in approximately \num{44} percent of storms, 
    post-thresholding.

In Figure~\ref{plot:marginal_pia}, we see the same marginal CDF's, for the Philadelphia International
    Airport.  Recall, this airport is situated along the banks of the Delaware River; storm surge
    has much further to travel to reach this point, and yet it experiences inundation much more 
    frequently, owing to that its elevation is only 4 feet above sea level, rather than the 9 feet
    of Dover AFB \makenote{[confirm]}.

\begin{table}[ht]
\centering
% latex table generated in R 4.4.1 by xtable 1.8-4 package
% Thu Oct 10 16:19:13 2024
\begin{table}[ht]
\centering
\begin{tabular}{lrrr}
  \hline
Model & 0.9 & 0.99 & 0.999 \\ 
  \hline
Monte Carlo & 14 & 19 & 19 \\ 
  Var Bayes & 5 & 5 & 6 \\ 
  Reg w/o FE & 81 & 106 & 116 \\ 
  Reg w/ FE & 86 & 113 & 123 \\ 
   \hline
\end{tabular}
\caption{Cluster concentration for identified models and fitting 
        methods, on the \emph{Restricted} Slice: columns specify quantiles 
        detailing the proportion of data contained within the table cells 
        indicated number of clusters.\label{tab:cluster_concentration}} 
\end{table}

\caption{Cluster concentration for identified models and fitting 
        methods, on the \emph{Restricted} Slice: columns specify quantiles 
        detailing the proportion of data contained within the table cells 
        indicated number of clusters.\label{tab:cluster_concentration}} 
\end{table}

Looking at equivalent marginal plots for all locations, nearly all preserve the
    same ordering, from top-left to bottom-right: first the Variational Bayes fit of PYPG, then the 
    Monte Carlo fit of the same model, then the regression models---though the specific ordering of
    the regression models changes.  The marginal empirical CDFs tends to bounce\makenote{[rephrase]}
    between the MCMC model, and the regression models.  This means that the variational fit
    tends to consistently predict lower values than is appropriate. This permits us some insight 
    to comment what effect granularity, or the number of extant clusters, has in model fidelity.
    In Table~\ref{tab:cluster_concentration}, we see detail the number of clusters, or Pitman-Yor
    process mixture components, identified in the dataset under the specified model and fitting 
    method.  Observe the large difference between the MCMC fitting method and the VB fitting
    method. 
    \makenote{table of average $\text{E}[\alpha_s]$ under each model, as well as 
    $\text{E}\left[\frac{\alpha_s}{\lVert\bm{\alpha}\rVert_{\infty}}\right]$}

\begin{table}[ht]
\centering 
% latex table generated in R 4.4.1 by xtable 1.8-4 package
% Thu Oct 10 16:54:40 2024
\begin{tabular}{lrrrr}
  \hline
Slice & Var Bayes & Monte Carlo & Reg w/o RE & Reg w/ RE \\ 
  \hline
  Threshold & 3 & 37 & ~ & ~ \\ 
  Delaware & 4 & 30 & ~ & ~ \\ 
  Restricted & 8 & 21 & 127 & 118 \\ 
  Critical & 11 & 51 & 30 & 22 \\ 
  \hline
\end{tabular}

\caption{Number of emergent clusters identified via posterior sampling.\label{tab:extant_clusters}}
\end{table}

We previously discussed Equation~\ref{eqn:condsurv}, which establishes the conditional probability of
    exceeding a specified threshold for some set of dimensions given other dimensions exceed their specified 
    threshold. Using the \emph{Critical} slice with a model fitted via MCMC, we use this equation to establish 
    conditional survival curves 
    for three locations: Dover Air Force Base, Philadelphia International Airport, as well as the Packer Avenue
    Marine Terminal, a major shipping hub.  Packer Avenue Terminal is only a few miles up the Delaware River from
    PIA, so we may expect a strong association in extreme behavior between those two locations.
    In keeping with our goal of a practical actionable metric, we 
    consider three scenarios where we observe extreme behavior further out in the bay than the positions of
    interest.  In the \emph{Lower Bay} scenario, we observe extreme behavior at sites on the south side of the 
    bay towards the entrance of the bay.  That is, a scenario in which sites 1,2, and 7 (Beebe Hospital, 
    Henlopen Memorial Park, and Smyrna Airport respectively) experienced storm surge at or above their
    respective 90th percentiles.  In the \emph{Upper Bay} scenario, we observe extreme behavior at sites 6,
    and 8 (Bay Island Fish and Wildlife Refuge, and Salem Airfield respectively), sites situated along the
    northern edge of Delaware Bay.  In the \emph{Mouth} scenario, we observe extreme behavior at all sites near
    the mouth of the bay.  This includes sites 1, 2, 3, and 4 (Beebe Hospital, Henlopen Memorial Park, 
    Paramount Airport, and the Cape Regional Medical Center). 

\subsection{Conditional Survival Curves}
\begin{figure}[htb]
    \begin{subfigure}[t]{0.31\textwidth}
        \centering
        \includegraphics[width=0.99\linewidth]{./plots/condsurv/doverafb}
        \caption{Dover Air Force Base\label{fig:condsurv1d:doverafb}}
    \end{subfigure}%
    ~ 
    \begin{subfigure}[t]{0.31\textwidth}
        \centering
        \includegraphics[width=0.99\linewidth]{./plots/condsurv/pia}
        \caption{Philadelphia International Airport\label{fig:condsurv1d:pia}}
    \end{subfigure}%
    ~
    \begin{subfigure}[t]{0.31\textwidth}
        \centering
        \includegraphics[width=\linewidth]{./plots/condsurv/packerave}
        \caption{Packer Avenue Marine Terminal\label{fig:condsurv1d:packerave}}
    \end{subfigure}
    \caption{Conditional survival curves for the labelled locations, under three scenarios of the observed \label{fig:condsurv1d}}
\end{figure}

Figure~\ref{fig:condsurv1d} shows the one-dimensional survival curves for these three locations, under 
    these three scenarios.  Note, the survival curve indicates $P(Z_s > z_s)$ given the stated scenario.
    Additionally, a $\bm{z}$ score greater than 1 indicates storm surge above the 90th percentile.
    Perhaps unsurprisingly in interpreting these results, as Dover AFB is on the south side of the bay, 
    we see the survival curve for the Upper Bay scenario dip below that of both Lower Bay and Mouth scenarios.  
    What is interesting, however, is that that ordering is not uniform; we see the ordering change to that
    behavior around $z = 4$.  The Mouth scenario indicates a storm that has inundated both the lower and
    upper portions of the Delaware Bay entrance, indicating an extremely powerful storm that is well positioned
    to enter the Bay.  As such, it is no surprise that the survival curve associated with that scenario
    indicates the highest probability of extreme surge throughout the entire curve.  What is interesting is that
    we observe the same crossing behavior and the same ordering on all three curves, though their exact shape
    and the exact point at which that cross occurs differ.  It is apparent that relative to the other
    scenarios, extreme surge on the Upper Bay sites does not strongly indicate increased surge at the other 
    sites.  It may be the case that a hurricane optimally aligned towards inundating the North bank is 
    sub-optimally aligned towards inundating the rest of the bay.

\begin{figure}[htb]
    \centering
    \includegraphics[width=0.3\linewidth]{./plots/condsurv/doverafb_pia}
    \caption{Conditional Survival Curve (Contour Plot) of flooding at Dover AFB (X) vs Philadelphia International 
        Airport (Y).\label{fig:condsurv2d:doverpia}
        \makenote{Add plot in real units}}
\end{figure}

\begin{figure}[htb]
    \centering
    \includegraphics[width=0.3\linewidth]{./plots/condsurv/doverafb_packer}
    \caption{Conditional Survival Curve (Contour Plot) of flooding at Dover AFB (X) vs Packer Avenue Terminal (Y).
        \label{fig:condsurv2d:doverpacker}
        \makenote{Add plot in real units}
        }
\end{figure}

Figure~\ref{fig:condsurv2d:doverpia} provides a contour plot of a two-dimensional survival surface, between
    flooding at Dover AFB and flooding at PIA, conditional on the listed scenarios.  Recalling that Dover AFB 
    is on the south edge of the bay, while PIA is far up the Delaware River, we may not expect to see a strong
    association between the two locations.  Indeed, on the contour plot we observe the survival surface is nearly 
    flat on the transition between Dover AFB and PIA. \makenote{This phrasing is terrible.}  This shape indicates
    an only moderate dependence between the locations, under these scenarios.  But we must call 
    attention to the ordering of the contours of the survival surface.  Here, the Lower Bay scenario crosses 
    even the Mouth scenario.  In contrast, in Figure~\ref{fig:condsurv2d:doverpacker}, we see that survival
    surface is actually concave, which indicates an extremely weak dependence between the two 
    locations.  This is understandable, as Packer Avenue Terminal is around 4 miles upstream of PIA,
    even further away from Dover AFB.    
    
\begin{figure}[htb]
    \centering
    \includegraphics[width=0.3\linewidth]{./plots/condsurv/pia_packer}
    \caption{Conditional Survival Curve (Contour Plot) of flooding at Dover AFB (X) vs Packer Avenue Terminal (Y). 
        \label{fig:condsurv2d:piapacker}
        \makenote{Add plot in real units}}
\end{figure}

Figure~\ref{fig:condsurv2d:piapacker} similarly presents a contour plot of the survival surface between 
    PIA and Packer Avenue terminal. Given the proximity of the locations, we would expect a strong 
    dependence.  That dependence is borne out, as in the contour plot of the survival surface we observe 
    strong convexity.


\makenote{Insert conditional survival curves given threshold under regression model.}

\makenote{insert posterior clustering discussion and pairs plot of cluster assignments of $\bm{\theta}$'s.}

% EOF