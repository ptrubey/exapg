\begin{comment}
    \begin{itemize}
        \item Description of joint maxima problem/application
        \item Estimation of the extremal dependence structure
        \begin{itemize}
            \item Following the example of \cite{trubey:pg}, ...
            \item Probability of extreme local inundation given current
                inundation field.
                \begin{equation*}
                    \begin{aligned}
                    P\left(Z_{s} \geq z_{s} \mid \bm{Z}_{\neg s} \geq \bm{z}_{\neg s}\right)
                        &= P\left(RV_{s} \geq rv{s} \mid RV_{\neg s} \geq rv_{\neq s}\right)\\
                        &= P\left(R\geq r\right)P\left(V_{s}\geq v_{s}\mid ...? \right)
                    \end{aligned}
                \end{equation*}
        \end{itemize}
        \item Multi-site return levels
        \begin{itemize}
            \item What is a multivariate return period/level?
            \begin{itemize}
                \item Conceptually it's not easy to define.  No two storms will present exactly
                    the same in magnitude or resulting scaled field.
                \item Marginally, \[T_p = \frac{\mu}{1 - F_x(X_p)}\] where $\mu$ is average inter-arrival 
                    in series of events.  Conceptual problem occurs in definition of $F$, which is unclear
                    in a multivariate setting.
                \item \url{https://agupubs.onlinelibrary.wiley.com/doi/full/10.1002/wrcr.20204}
                \item \url{https://agupubs.onlinelibrary.wiley.com/doi/full/10.1029/2009WR009040}
                \item \url{https://hess.copernicus.org/articles/17/1281/2013/}
                \item \url{https://nhess.copernicus.org/articles/12/2699/2012/}
                
                \item need to make a statement harkening back to the fact this is a simulation of
                    storm parameters sampled via latin hypercube, so the definition of 'return period'
                    in this context is suspect.
            \end{itemize}
            \item Given shaky conception of multivariate return period, we might consider alternative 
                definitions more relevant to an inundation field.
            \begin{itemize}
                \item Total Inundation field?  (some univariate transformation?)
                \item number of locales under significant flooding?
            \end{itemize}
        \end{itemize}
        \item Application results
    \end{itemize}
\end{comment}

\subsection{Applications of Extremal Dependence}
Following the example of \cite{trubey:pg}, we can use the dependence structure we infer
    by fitting the model in Equation~\eqref{eqn:pypg}.  Then, conditional on



The standard Pareto distribution exhibits a multiplicative memory-less property such that
\[
\text{P}(Z > st \mid Z > s) = P(Z > t)\text{ for }s,t \geq 1
\]

\subsection{Multi-site Return Periods}
Statistical inference of extreme values tends to focus on the \emph{return period} as a deliverable
    metric.  In a univariate case, this is relatively easy to define: for event $z$, the return period
    is the average \emph{time} it would take to observe a new event $Z$ as extreme or more extreme than $z$.
    that is,
    \[
    T(z) = \frac{\mu}{1 - F_z(z)}
    \]
    where $\mu$ is the average inter-arrival time in a series of events. Conceptual problems begin to arise
    in interpretation when we consider a multivariate $F$. 
    \makenote{extreme analysis book makes succinct argument; quote.}
    \cite{cho2023} avoids this issue of interpretation by estimating univariate return periods for each
    indexed location within a spatial field, along with low-dimensional multivariate extreme analysis 
    on different summary statistics, assessed over the aggregate spatial field: flood volume, 
    peak discharge, total rainfall depth, and maximum wind speed.
    A more analogous approach to ours, in that they consider the dependence structure of extreme values of 
    the same summary statistic, between different points in space may be found in \cite{salvadori2010}.  They
    consider yearly maximum observed flow rates between 4 of 17 available flow meters on a river catchment
    in Britain. \makenote{
        Inference task size: 4 locations $\times$ 37 years available.  I suspect the problem,
        or why they're not more ambitious in terms of scale, is one of interpretation: more than 4 dimensions 
        makes presenting results difficult.  Also, lack of data (\# of years).
        }
    Beyond the conceptual issue of interpreting a multivariate $F$, there arises a practical issue in presenting
    complete results of a higher dimensional process.
       
\subsection{Conditional Survival}
One of the more compelling applications of modelling the dependence structure between locations in 
    storm surge lies in the reality that a storm surge occurs over a period of time, and the 
    maximum observed values in storm surge at different sites occur asynchronously.  A decision maker,
    interested in the storm surge at a smaller group of locations, can observe storm surge at other
    locations, and make an inference about the probability of catastrophic flooding at their locations
    of interest and make an informed decision.
    Equation~\eqref{eqn:condsurv}, from Proposition~2 of \cite{trubey:pg} offers us a means by which 
    this can be accomplished.
    \begin{equation}
        \label{eqn:condsurv}
        \text{Pr}\left[\bigcap_{s\in\alpha}Z_s \geq z_s\;\middle|\;\bigcap_{s\not\in\alpha}Z_s \geq z_s\right] =
        \frac{
            \text{E}\left[\bigwedge_{s=1}^S 1\wedge \frac{V_s}{z_s}\right]
        }{
            \text{E}\left[\bigwedge_{s\not\in\alpha} 1\wedge \frac{V_s}{z_s}\right]
        }
    \end{equation}
    Letting Set $\alpha$ be a group of locations of interest, we can establish the probability of
    entering a failure region, conditional on the current state and potential future state of the inundation 
    field.  Say, given current storm flooding near the mouth of the Delaware Bay, will the Philadelphia 
    International Airport, situated on the Delaware river, experience catastrophic flooding?  If one can 
    describe the dependence structure of extremes in inundation, Equation~\ref{eqn:condsurv} offers a 
    practical, actionable metric.

    \makenote{Select a few locations of interest in each inference task; establish conditional survival 
    probabilities for each given a few }
    
    The framework of multivariate EVT does not allow for the concept of negative dependence.  In fact, 
    a peaks-over-threshold approach using the projected gamma to model the dependence structure can not 
    encompass even complete independence: the closest we can envision is a weak positive dependence.
    As such, it is difficult to accurately model phenomena that 
    Never the less, \makenote{making a statement about the probability of simultaneous flooding in new york and virginia.}

    






% EOF