\begin{comment}
    \begin{itemize}
        \item Description of joint maxima problem/application
        \item Estimation of the extremal dependence structure
        \begin{itemize}
            \item Following the example of \cite{trubey:pg}, ...
            \item Probability of extreme local inundation given current
                inundation field.
                \begin{equation*}
                    \begin{aligned}
                    P\left(Z_{s} \geq z_{s} \mid \bm{Z}_{\neg s} \geq \bm{z}_{\neg s}\right)
                        &= P\left(RV_{s} \geq rv{s} \mid RV_{\neg s} \geq rv_{\neq s}\right)\\
                        &= P\left(R\geq r\right)P\left(V_{s}\geq v_{s}\mid ...? \right)
                    \end{aligned}
                \end{equation*}
        \end{itemize}
        \item Multi-site return levels
        \begin{itemize}
            \item What is a multivariate return period/level?
            \begin{itemize}
                \item Conceptually it's not easy to define.  No two storms will present exactly
                    the same in magnitude or resulting scaled field.
                \item Marginally, \[T_p = \frac{\mu}{1 - F_x(X_p)}\] where $\mu$ is average inter-arrival 
                    in series of events.  Conceptual problem occurs in definition of $F$, which is unclear
                    in a multivariate setting.
                \item \url{https://agupubs.onlinelibrary.wiley.com/doi/full/10.1002/wrcr.20204}
                \item \url{https://agupubs.onlinelibrary.wiley.com/doi/full/10.1029/2009WR009040}
                \item \url{https://hess.copernicus.org/articles/17/1281/2013/}
                
                \item need to make a statement harkening back to the fact this is a simulation of
                    storm parameters sampled via latin hypercube, so the definition of 'return period'
                    in this context is suspect.
            \end{itemize}
            \item Given shaky conception of multivariate return period, we might consider alternative 
                definitions more relevant to an inundation field.
            \begin{itemize}
                \item Total Inundation field?  (some univariate transformation?)
                \item number of locales under significant flooding?
            \end{itemize}
        \end{itemize}
        \item Application results
    \end{itemize}
\end{comment}

Following the example of \cite{trubey:pg}, we 



\begin{equation}
    \text{Pr}\left[\bigcap_{s\in\alpha}Z_s > z_s\;\middle|\;\bigcap_{s\not\in\alpha}Z_s > z_s\right] =
    \frac{
        \text{E}\left[\bigwedge_{s=1}^S 1\wedge \frac{V_s}{z_s}\right]
    }{
        \text{E}\left[\bigwedge_{s\not\in\alpha} 1\wedge \frac{V_s}{z_s}\right]
    }    
\end{equation}






% EOF