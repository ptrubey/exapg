Following the example of \cite{trubey:pg}, we can use the dependence structure we infer
    by fitting the model in Equation~\eqref{eqn:pypg}.  Then, conditional on the fitted model,
    there are various inferences we can make.

\subsection{Multi-site Return Periods and Conditional Survival}
Statistical inference of extreme values tends to focus on the \emph{return period} as a deliverable
    metric.  In a univariate case, this is easy to define: for event $z$, the return period
    is the average \emph{time} it would take to observe a new event $Z$ as extreme or more extreme than $z$.
    that is,
    \[
    T(z) = \frac{\mu}{1 - F_z(z)} = \frac{\mu}{S_z(z)}
    \]
    where $\mu$ is the average inter-arrival time in a series of events, and $S(\cdot)$ the survival function. 
    Conceptual problems begin to arise in interpretation when we consider a multivariate $F$.  
    \cite{salvadori2010} considers a strict interpretation of the return period in a multivariate setting,
    defining the return period in terms of a copula---a redefinition of the joint CDF, as a distribution
    over marginal uniform densities, with the marginal CDF's for each site taking the place of the marginal 
    uniforms.  We say strict interpretation in interpretation of the CDF: 
    $F(\bm{z}) = \text{P}\left(\cap_{s \in S}Z_s\leq z_s\right)$, though it considers later a more general
    \emph{critical region}.
    \cite{Salvadori2013} inverts this, defining the return period in terms of the joint survival function, $T(\bm{z}) = \frac{\mu}{S(\bm{z})}$, where 
    $S(\bm{z}) = \text{P}\left(\bigcap_{s\in S}Z_s > z_s\right)$.
    We observe there exists a vast difference in interpretation between these two extremes.  In generating a
    meaningful deliverable metric of a return period, for a given event at how many sites
    must the threshold be breached for us to interpret the event as over-topping the threshold?  For a CDF
    based metric, one; for a survival function based metric, all.  But as the number of locations under 
    consideration increases, the value of such a metric decreases, as the probability of the specified scenario
    approaches one or zero respectively.
    \cite{cho2023} sidesteps this issue of interpretation by estimating univariate return periods for each
    indexed location within a spatial field, along with low-dimensional multivariate extreme analysis 
    on different summary statistics, assessed over the aggregate spatial field: flood volume, 
    peak discharge, total rainfall depth, and maximum wind speed.  \cite{graler2013} also follows the latter 
    approach, establishing a multivariate return period using a copula over three dimensions of summary statistics.
    \cite{salvadori2010} follows an approach closer to ours, in that they consider the dependence 
    structure of extreme values of the same summary statistic between different points in space.  They
    consider yearly maximum observed flow rates between 4 of the 17 available flow meters on the catchment 
    of the river Spey, in Scotland. \makenote{As to why only 4 locations were considered, we suspect to be an issue of 
    interpretation: the greater the number of sites considered, the greater the delta between a calculable 
    result and the meaningfulness of the return period as a deliverable metric.} 
    Beyond the conceptual issue of interpreting a multivariate $F$, there arises a practical issue in presenting
    complete results of a higher dimensional process.
    It is for these reasons that in practice, as a deliverable metric, the notion of a return period is
    frequently tailored to the application in question.  In this paper, we tailor the survival function
    thresholds to describe specific scenarios. \bruno{\bf Do you actually use all this? I think you should just say 
    what we do, which is to consider the conditional joint probabilities, and mention this in passing.}

One of the more compelling applications of modeling the dependence structure between locations in 
    storm surge lies in the reality that a storm surge occurs over a period of time, and the 
    maximum observed values in storm surge at different sites occur asynchronously.  A decision maker,
    interested in the storm surge at a smaller group of locations, can observe storm surge at other
    locations, and make inference as to the probability of catastrophic flooding at their locations
    to make an informed decision.
    Equation~\eqref{eqn:condsurv}, from Proposition~2 of \cite{trubey:pg} offers us a practical means
    by which this can be accomplished.
    \begin{equation}
        \label{eqn:condsurv}
        \text{Pr}\left[\bigcap_{s\in\alpha}Z_s \geq z_s\;\middle|\;\bigcap_{s\not\in\alpha}Z_s \geq z_s\right] =
        \frac{
            \text{E}\left[\bigwedge_{s=1}^S 1\wedge \frac{V_s}{z_s}\right]
        }{
            \text{E}\left[\bigwedge_{s\not\in\alpha} 1\wedge \frac{V_s}{z_s}\right]
        }
    \end{equation}
    Letting Set $\alpha$ be a group of locations of interest, we obtain the probability of
    entering a failure region, conditional on the current state of the inundation field.  
    Say, given current storm flooding near the mouth of the Delaware Bay, will the 
    Philadelphia International Airport, situated on the Delaware river, experience catastrophic flooding?  
    If one can describe the dependence structure of extremes in inundation analytically or via samples,
    Equation~\ref{eqn:condsurv} offers a practical, actionable metric.

    One point of concern: the framework of multivariate EVT does not allow for the concept of negative dependence.  
    In fact, a peaks-over-threshold approach using the projected gamma to model the dependence structure can not
    encompass even complete independence: the closest we can represent is weak positive dependence.
    As such, it is difficult to accurately model phenomena that exhibit mutually exclusive extreme behavior.

    

% EOF