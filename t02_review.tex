Our analysis of the SLOSH data's extremal dependence requires some background.
    \makenote{introduce the review section.  avoid naked sections.}

\subsection{Extreme Value Theory\label{ref:evt}}
\makenote{first paragraph basically copy-pasted from ndpg.  Rephrase more?}
Extreme value theory uses the limited information available in a sample to
    characterize the tails of a distribution.  In the univariate case, asymptotic
    results provide a unique parametric limiting family for the maximum observation
    in a sample.  For data with such a limiting distribution, a common approach
    is to consider observations in excess of a high threshold, and model the
    excesses over that threshold under a generalized Pareto framework.  This
    approach is known as \emph{peaks over threshold} (PoT).  In the multivariate 
    case, the theory for PoT is well established (see, for example, 
    \cite{dehaan2006}), and it indicates the existence of a limiting distribution
    with no parametric representation.  This can pose a challenge for inference,
    but we are not without options.

The multivariate PoT model considered in this paper has been developed 
    in~\cite{trubey:pg}, based on a definition of the limiting distribution proposed
    proposed in~\cite{rootzen2018}.  
    Let $\bm{W} = (W_1,\ldots,W_d)$ be a $d$--dimensional random vector with 
    cumulative distribution $F$.  Assuming there exists a sequence of vectors
    $\bm{a}_n$, $\bm{b}_n$, and a $d$--variate distribution $G$ such that 
    $\lim\limits_{n\to\infty}F^n(\bm{a}_n\bm{w} + \bm{b}_n) = G(\bm{w})$, then
    $G$ is a $d$--variate generalized extreme value distribution.  Then,
    \[
        \lim\limits_{n\to\infty}\text{Pr}
            \left[\bm{a}_n^{-1}(\bm{W} - \bm{b}_n) 
                \leq \bm{w}\mid \bm{W}\not\leq \bm{b}_n\right]
        = \frac{\log G(\bm{w}\wedge \bm{0}) - \log G(\bm{w})}{\log G(\bm{0})}
        = H(\bm{w})
    \]
    where $H$ is the multivariate Pareto distribution.  \cite{rootzen2018}
    provides a number of stochastic representations of $H$, and in particular,
    Remark~1 justifies the representation given in \cite{ferreira2014} where,
    in the limit, we factorize $\bm{W} = R\bm{V}$ with $R$ and $\bm{V}$ independent.
    \makenote{I'm not very fond of this transition.  We standardize $\bm{W}$ to 
    $\bm{Z}$, then factorize $\bm{Z} = R\bm{V}$.  Separability holds true in both
    cases (i think), but is still dependent on marginal distributional parameters.
    standardization allows those to be ignored.  We should mention standardization
    before factorization.}
    $R = \lVert \bm{W}\rVert_{\infty}$ is distributed as a standard Pareto random
    variable, and $\bm{V} = \bm{W} / \lVert \bm{W}\rVert_{\infty}$ is a random
    vector existing within $\mathbb{S}_{\infty}^{d-1}$, the positive orthant of
    the unit sphere under the $\mathcal{L}_{\infty}$ norm.  $R$ and $\bm{V}$
    respectively comprise the \emph{radial} and \emph{angular} components of $H$.
    As $R$ and $\bm{V}$ are independent,  the distribution of $\bm{V}$ is 
    effectively the dependence structure of $\bm{W}$.

Let the threshold $b_{q\ell} = \hat{F}_{\ell}^{-1}(1 - 1/q)$, where $\hat{F}$ is
    the empirical cumulative distribution function for the $\ell$th component.
    Marginally, values exceeding the threshold $b_{q\ell}$ are assumed to follow
    a univariate generalized Pareto distribution, and are used to estimate the
    corresponding marginal scale and shape parameters $a_{\ell}$ and $\xi_{\ell}$
    respectively.  Setting $\bm{b}$ and having inferred $\bm{a}$, and $\bm{\xi}$, 
    we transform $\bm{w}\mid \bm{w}\not\leq \bm{b}$ to a standard multivariate 
    Pareto form via the transformation
    \begin{equation}
        z_{i\ell} = \left(1 + \xi_{\ell}\frac{w_{i\ell} 
            - b_{\ell}}{a_{\ell}}\right)_{+}^{1 / \xi_{\ell}}
    \end{equation}
    where $(\cdot)_+$ indicates the positive parts function.  Let 
    $r_i = \lVert \bm{z}_i$, and $\bm{v}_i = \bm{z}_i / r_i$.  Due to 
    thresholding, $i$ ranges from 1 to $m\leq n$, and $r_i > 1$.  Recall that
    $R\in (1, \infty)$ will by design follow a standard Pareto distribution,
    the inferential task left to us is describing the distribution of the
    angular component $\bm{V}\in\mathbb{S}_{\infty}^{d-1}$.  

\subsection{Projected Gamma\label{ref:pg}}
A suitable distribution for $\bm{V}$ can be approximated by projecting a 
    distribution in $\mathbb{R}_+^d$ onto $\mathbb{S}_{p}^{d-1}$.  
    Recall that the $\mathcal{L}_p$ norm 
    $\lVert \bm{s}\rVert_p = \left(\sum_{\ell = 1}^d\right)^{1/p}$.  Then
    for $\bm{x}\in\mathbb{R}_+^d$, we define the transformation
    \begin{equation}
        \label{eqn:projection}
        T_p(\bm{x}) = \left(\lVert \bm{x}\rVert_p, 
            \frac{x_1}{\lVert \bm{x}\rVert_p},\ldots, 
                \frac{x_{d-1}}{\lVert \bm{x}\rVert_p}\right)
                =: (r,\bm{y})
    \end{equation}
    where $y = (y_1,\ldots,y_{d-1}) \in \mathbb{S}_{p}^{d-1}$, $r > 0$, and 
    $y_d = \left(1 - \sum_{\ell = 1}^{d-1}y_{\ell}^p\right)^{\frac{1}{p}}$.
    By transforming $\bm{x}$ to $(r,\bm{y})$ and integrating out $r$, we
    succeed in establishing a distribution for $\bm{y}$ on 
    $\mathbb{S}_{\infty}^{d-1}$
    The Jacobian of the transformation in Equation~\eqref{eqn:projection} is
    $r^{d-1}\left[Y_d + \sum_{\ell = 1}^{d-1}y_{\ell}^py_d^{1-p}\right]$.
    This Jacobian is well suited to a product of gammas density, where 
    $f(\bm{x}\mid\bm{\alpha},\bm{\beta}) = 
        \prod_{\ell = 1}^d\mathcal{G}(x_{\ell}\mid\alpha_{\ell},\beta_{\ell})$.
    Transforming $\bm{x}$ as described, $r$ can be integrated out in closed
    form, and we are left with the \emph{projected gamma} density for arbitrary 
    $p > 0$.
    \[
        f(\bm{y}\mid\bm{\alpha},\bm{\beta}) = \prod_{\ell = 1}^d\left[
            \frac{\beta_{\ell}^{\alpha_{\ell}}}{\Gamma(\alpha_{\ell})}
            y_{\ell}^{\alpha_{\ell} - 1}\right]
            \left[y_d + \sum_{\ell = 1}^{d-1}y_{\ell}^py_d^{1-p}\right]
            \frac{\Gamma(\sum_{\ell = 1}^d \alpha_{\ell})}{\left(
                \sum_{\ell = 1}^d\beta_{\ell}y_{\ell}
                \right)^{\sum_{\ell = 1}^d \alpha_{\ell}}
            }
    \]
    Of course, at $p=\infty$, the transformation is not differentiable, so a 
    direct projection onto $\mathbb{S}_{\infty}^{d-1}$ is not possible. We
    instead desire a high but finite $p$.
    Here we balance two issues: as $p\to\infty$, the space $\mathbb{S}_{p}^{d-1}$ 
    will approach asymptotically towards $\mathbb{S}_{\infty}^{d-1}$.
    Unfortunately, also as $p\to\infty$, the stability of the Jacobian of the
    transformation becomes increasingly dependent on the value, or choice, of 
    $y_d$.  If $y_d\to 0$, then the Jacobian diverges, and the distribution 
    becomes numerically unstable. We select $p = 10$ to balance these concerns.

To better capture the variability in the data, we use the projected gamma density 
    as a kernel density of an Bayesian non-parametric mixture model, based on 
    the Pitman-Yor process introduced in~\cite{perman1992}.  A Pitman-Yor process
    is a fully atomic measure specified by two parameters and a centering
    distribution.  Under a stick-breaking representation as described 
    in~\cite{ishwaran2001}, 
    \[
        \text{Pr}(\bm{\alpha}\mid\ldots) 
            = \sum_{j = 1}^Jp_j\delta_{\bm{\alpha}_j};\;\;\;
            \sum_{j=1}^Jp_j = 1;\;\;\;
            p_j := \chi_j\prod_{k = 1}^{j-1}(1 - \chi_k)
    \]
    where $\delta_{\bm{\alpha}_j}$ indicates a point mass at $\bm{\alpha}_j$ and
    $\bm{\alpha}_j$ are sampled independently from the centering distribution $G_0$.
    Under the Pitman-Yor process, $\chi_j \sim \text{Beta}(1 - d, \eta + jd)$
    where $d \in [0, 1)$ denotes the \emph{discount} parameter and $\eta > -d$
    denotes the \emph{concentration} parameter.

The projected gamma based model is not an inherently spatial model.  As such, 
    rather than holding the entire grid in memory, it makes sense to consider 
    data pertaining only to landmarks of interest.    If one is performing 
    contingency planning in preparation for the storm season, it would be 
    helpful to know the probability of such services being rendered inoperable 
    simultaneously, precisely when they are needed most.  If one is seeking 
    sites for a new service provider, it would be helpful understand the 
    likelihood of a proposed location being rendered inoperable in the same 
    manner.  For these reasons, and considering the computational complexity of 
    a 23 million dimensional model, rather than consider the entire grid of data 
    pertaining to any particular storm, we subset the data to grid cells in the 
    vicinity of such locations of interest.  These locations are gathered from 
    the 2023 US Census Bureau's \emph{TIGER} database \needcite and specifically 
    the point and landmark file, and only grid cells that experienced \emph{some} 
    inundation in greater than 2 \makenote{(subject to change)} percent of storm 
    simulations.   This last requirement arises as consequence of fitting the 
    generalized Pareto: the threshold for excesses $b$ needs to be set above the 
    minimum observed value $(0)$ of the dataset.  
    \makenote{(idea: zero-inflated projected gamma?)}  \makenote{Probably not
    necessary...} \st{We identify grid cells in}
    \st{the vicinity of such locations by establishing a small buffer around said}
    \st{locations, and spatially intersecting the buffer with the grid.}

\subsection{Variational methods: a brief overview\label{ref:varbayes}}
\makenote{These 2 paragraphs are intended to introduce the justification for 
    VarBayes. I like the wordsmithing at the end of the second paragraph, but 
    I'm unsure of the necessity.  Specifically, the necessity of a generic 
    VarBayes intro as detailed here.}
Let $\bm{x}$ be the observed data, and $\bm{\theta}$ be an unobserved set of 
    parameters governing the distribution of the observed data.  We term 
    $f(\bm{x}\mid\bm{\theta})$ the likelihood.  The goal of inference in 
    general is to obtain information about $\theta$, using information 
    obtained from $\bm{x}$. Bayesian inference in particular places a prior
    distribution on $\theta$, $f(\bm{\theta})$, and from this we can obtain a
    posterior distribution on $\bm{\theta}$ that incorporates both information
    from the likelihood as well as the prior.  That is,
    \[
        f(\bm{\theta}\mid\bm{x}) = 
            \frac{f(\bm{x}\mid\bm{\theta})f(\bm{\theta})}{f(\bm{x})},
    \]
    where the demoninator $f(\bm{x})$ is obtained as
    \[
        f(\bm{x}) = \int_{\Omega_{\bm{\theta}}}f(\bm{x}\mid\bm{\theta})f(\bm{\theta})d\bm{\theta}
    \]
    where $\Omega_{\bm{\theta}}$ denotes the support of $\bm{\theta}$.

When a given model is sufficiently complex that the above integration is no
    longer feasible analytically, or if the posterior cannot be described in
    closed form\makenote{when the posterior is not tractible...}, we frequently 
    turn to sampling--based classes of methods, such as 
    \emph{Markov chain Monte Carlo} (MCMC)\needcite. There are myriad approaches, 
    \makenote{expand?}, but central to all sampling--based 
    methods is the stochastic nature of model fitting, and the 
    necessity of samples for posterior inference.  Sampling methods can be 
    tempermental, requiring many numbers of iterations to reach convergence.
    Samples take time to draw, and memory to store; but for sufficiently large 
    problems, both can be in short supply.

One class of methods that shows promise in avoiding the computational and 
    storage burdens of sampling methods is variational inference.\needcite 
    Based on variational calculus, this approach attempts to approximate the 
    true posterior distribution
    $f(\bm{\theta}\mid\bm{x})$ using a tractible \emph{surrogate posterior} 
    $q(\bm{\theta})$.  Model fitting under this class involves specification of 
    the family of surrogate densities $\mathcal{Q}$, then selecting that density 
    $q^*$ which minimizes the \emph{Kullbeck-Liebler} (KL) divergence between 
    the surrogate and true posterior.  That is,
    \begin{equation}
        \label{eqn:optimalq}
        q^*(\bm{\theta}) = \argmin_{q\in\mathcal{Q}}\left\lbrace
        \text{KL}\left(q(\bm{\theta})||f(\bm{\theta}\mid\bm{x})\right) 
        \coloneqq
        \text{E}_{q}\left[\log\left(
        \frac{q(\bm{\theta})}{f(\bm{\theta}\mid\bm{x})}
        \right)\right]
        \right\rbrace.
    \end{equation}
    Using KL divergence as an analogue to distance, we are attempting to find
    the \emph{closest} tractable surrogate posterior $q$ within the family 
    $\mathcal{Q}$ to the true posterior.
    As has been explained however, we do not necessarily have 
    $f(\bm{\theta}\mid\bm{y})$ directly.  Let 
    $f(\bm{y},\bm{\theta}) = f(\bm{y}\mid\bm{\theta})f(\bm{\theta})$.
    Then we reformulate the KL divergence into
    \[
        \begin{aligned}
        \text{KL}\infdiv{q(\bm{\theta})}{f(\bm{\theta}\mid\bm{x})}
            &=\text{E}_{q}\left[\log\left(
                \frac{q(\bm{\theta})}{f(\bm{\theta}\mid\bm{x})}
                \right)\right]\\
            &= \text{E}_{q}\left[
                \frac{q(\bm{\theta})f(\bm{x})}{f(\bm{y},\bm{\theta})}
                \right]\\
            &= \text{E}_{q}\left[
                \log q(\bm{\theta}) - \log f(\bm{x},\bm{\theta}) 
                + \log f(\bm{x})
                \right]\\
            &= \text{E}_{q}\left[
                \log q(\bm{\theta}) - \log f(\bm{x},\bm{\theta})\right] + 
                   \log f(\bm{x}).
        \end{aligned}
    \]
    Note that the second term, the \emph{evidence}, is constant with respect to 
    $\bm{\theta}$.  With a bit of rearrangement,
    \[
        \begin{aligned}
        \log f(\bm{x}) 
            &= \text{KL}\infdiv{q(\bm{\theta})}{f(\bm{\theta}\mid\bm{x})}
                - \text{E}_q\left[
                \log q(\bm{\theta}) - \log f(\bm{x},\bm{\theta})
                \right]\\
            &= \text{KL}\infdiv{q(\bm{\theta})}{f(\bm{\theta}\mid\bm{x})}
                + \mathcal{L}(\bm{\theta})
        \end{aligned}
    \]
    Thus maximizing $\mathcal{L}(\bm{\theta})$ is equvivalent to minimizing the 
    KL divergence. In this form, as KL divergence is a positive quantity, it is
    readily apparent that $\mathcal{L}(\bm{\theta})$ establishes a lower bound
    on the evidence.  As such, the variational literature has christened it the
    \emph{evidence lower bound}, frequently abbreviated to \emph{ELBO}.  As such,
    we can rewrite Equation~\eqref{eqn:optimalq} to
    \[
        q^*(\bm{\theta}) = \argmax_{q\in\mathcal{Q}} \mathcal{L}(\bm{\theta})
    \]
    
\subsection{Optimizing the ELBO: Automatic Differentiation}
    




% EOF 