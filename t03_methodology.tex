$\bm{Z} = R\bm{V}$, where $\bm{V} \in \mathbb{S}_{\infty}^{d-1}$; 
$R \in \mathbb{R}_+$; $R$ and $\bm{V}$ are independent.  
$\bm{Y} = T_p(\bm{V})$ is the normalization of $\bm{V}$ onto 
$\mathbb{S}_p^{d-1}$.

\subsection{Target Distribution (gamma prior)}
\begin{equation}
    \begin{aligned}
        \bm{y}_i \mid \bm{\alpha}_i &\sim
        \mathcal{PG}_p\left(\bm{Y}\mid\bm{\alpha}_i,\bm{1}\right)\\
        \bm{\alpha}_i &\sim G\\
        G &\sim \mathcal{PY}\left(\eta, d, G_0\right)        
    \end{aligned}
    ~\hspace{1cm}
    \begin{aligned}
        G_0 &= {\textstyle\prod}_{\ell = 1}^{d}\mathcal{G}(\alpha_{\ell}\mid \xi_{\ell},\tau_{\ell})\\
        \xi_{\ell} &\sim \mathcal{G}(\xi\mid a, b)\\
        \tau_{\ell} &\sim \mathcal{G}(\tau\mid c, d)
    \end{aligned} 
\end{equation}
If we expand the Pitman--Yor process using stick-breaking notation with truncation point $J$, we can write the above as
\begin{equation}
    \begin{aligned}
        \nu_j\mid\eta &\sim \mathcal{B}(1 - d, \eta + kd)\\
        \bm{\pi} &= \pi(\bm{\pi})\\
        \alpha_{j\ell} &\sim \mathcal{G}(\alpha\mid\xi_{\ell},\tau_{\ell})
    \end{aligned}
    ~\hspace{1cm}
    \begin{aligned}
        \xi_{\ell} &\sim \mathcal{G}(\xi \mid a, b)\\
        \tau_{\ell} &\sim \mathcal{G}(\tau\mid c, d)\\
        \bm{y}_i \mid \ldots &\sim {\textstyle\sum}_{j = 1}^Jw_i\mathcal{PG}(\bm{y}\mid\bm{\alpha}_j, \bm{1})
    \end{aligned}   
\end{equation}
where $\pi(\bm{\nu})$ is a function of $\bm{\nu}$ such that
\begin{equation*}
    \bm{\pi} = \pi(\bm{v}) = \begin{cases}
        \nu_j &\text{ for }j = 1\\
        \nu_j{\textstyle\prod}_{k < j}(1 - \nu_k) &\text{ for }j = 2,\ldots,J-1\\
        {\textstyle\prod}_{k < j}(1 - \nu_k) &\text{ for }j = J
    \end{cases}
\end{equation*}
Fitting the above model is accomplished using a variational approximation.  As a first pass, we attempt a simple surrogate posterior using a mean field representation.

\subsubsection{Gaussian Mean Field Variational Family surrogate posterior for Pitman-Yor mixture of projected gammas, gamma prior}
The mean field surrogate posterior creates an independent distribution for every parameter of the model, ideally with the appropriate transformation
for each parameter such that the support of the transformed variable is the real line.   Optimization through gradient ascent makes discrete parameters ineligible
for fitting through a variational approach, so we ascertain cluster weights globally rather than sampling the underlying cluster membership.
\begin{equation}
    \begin{aligned}
    \bm{\nu} &\sim {\textstyle\prod}_{j = 1}^J\text{Logit}\mathcal{N}(\nu_j\mid\mu_{\nu_j},\sigma^2_{\nu_j})\\
    \bm{\alpha} &\sim {\textstyle\prod}_{j = 1}^J{\textstyle\prod}_{\ell = 1}^D\mathcal{LN}(\alpha_{j\ell}\mid\mu_{\alpha_{j\ell}},\sigma^2_{\alpha_{j\ell}})
    \end{aligned}
    ~\hspace{1cm}
    \begin{aligned}
    \bm{\xi} &\sim{\textstyle\prod}_{\ell = 1}^D\mathcal{LN}(\xi_{\ell}\mid\mu_{\xi_{\ell}},\sigma^2_{\xi_{\ell}})\\
    \bm{\tau} &\sim{\textstyle\prod}_{\ell = 1}^D\mathcal{LN}(\tau_{\ell}\mid\mu_{\tau_{\ell}}, \sigma^2_{\tau_{\ell}})
    \end{aligned}
\end{equation}
We fit this surrogate posterior in \emph{tensorflow} using automatic differentiation to calculate the gradient, and 

\subsection{Variational Approximation with exact sampling}
This variational approximation makes use of the exact sampling of latent variables as detailed in \cite{Loaizamaya2022} to
sample cluster membership, and thus cluster weights, via their full conditional distributions.  Then, conditional on
the sampled cluster weights/memberships, a simpler variational approximation is available for cluster and prior parameters.

Under the stick-breaking representation, cluster membership is distributed categorically as
\begin{equation}
    \text{P}(\gamma_i = j) \propto \pi_j\mathcal{PG}(\bm{y}_i\mid\bm{\alpha}_j,\bm{1})
\end{equation}
with the appropriate normalizing constant.  Then, the raw cluster weight $\nu_j$ is distributed as
\begin{equation}
    \nu_j\mid\bm{\gamma} \sim \text{Beta}(\eta - d + n_j,1 + jd + \sum_{k > j}n_k)
\end{equation}
where $n_j = \sum_{i:\gamma_i = j}1$.  Then $\bm{\pi} = \pi(\bm{\nu})$.

Let $\theta = (\bm{\alpha},\bm{\xi},\bm{\tau})$ be a subset of parameters for which the distribution will be approximated via variational inference,
and $\phi = (\bm{\gamma},\bm{\nu})$ be the latent parameters that will be sampled via their full conditional distributions.  Then,
\begin{equation}
    \begin{aligned}
        f(\bm{y},\theta\mid\phi) &\propto \mathcal \prod_{j = 1}^J\left[\prod_{i:\gamma_i = j}\left[\mathcal{PG}(\bm{y}_i\mid\bm{\alpha}_j,\bm{1})\right]\times\prod_{\ell = 1}^d\mathcal{G}(\alpha_{j\ell}\mid\xi_{\ell},\tau_{\ell})\right] \\
        &\hspace{1cm}\times \left[\prod_{\ell = 1}^d\mathcal{G}(\xi_{\ell}\mid a, b)\mathcal{G}(\tau_{\ell}\mid c,d)\right]        
    \end{aligned}
\end{equation}

\cite{tran2021}

\subsection{Clustering of storm parameters based on angular distribution}

One potential application leverageing use of a Bayesian non-parametri prior is 
    exploiting the clustering inherent to the method. Recall $\delta_i$ is the 
    cluster identifier for observation $i$.  Its sampling is made explicit in
    the MCMC model, though it is integrated out in the variational model. In
    posterior analysis, given $\pi$, where $\pi_j = \nu_j\prod_{k = 1}^{j-1}(1 - \nu_k)$
    and $\bm{\alpha}$, a posterior probability of cluster assignment even
    in the variational model can be sampled as
    \begin{equation}
        \label{eqn:clusterprob}
        \text{P}\left(\delta_i = j\mid\bm{\alpha},\bm{\nu},\bm{y}_i\right) 
            = \frac{\pi_j\mathcal{PG}(\bm{y}_i\mid\bm{\alpha}_j,\bm{1})}{
            \sum_{k = 1}^J \pi_j\mathcal{PG}(\bm{y}_i\mid\bm{\alpha}_k,\bm{1})}.
    \end{equation}
    This approach relies on samples of $\bm{\alpha},\bm{\nu}$ from the fitted
    model to generate \emph{loose} clusters for which there is no inherent 
    labelling.  This presents a problem, as interpreting clusters first requires
    \emph{labelling}, or a fixed assignment of observations to clusters.

There are multiple options for how cluster labelling can be accomplished.  A
    simple option migbht be to take
    \[
        \bm{\alpha}_j^* = \frac{1}{S}\bm{\alpha}_{js},\;\;\;\;
        \bm{\nu}_j^* = \frac{1}{S}\nu_{js}
    \]
    the posterior mean of cluster parameters and cluster weights, then 
    $\delta_i^{*} = \argmax P(\delta_i = j\mid\bm{\alpha}^*,\bm{\nu}^*,\bm{y}_i)$
    as defined in Equation~\ref{eqn:clusterprob}.  This option is more easily
    accomplished using the variational model, as it avoids the label-switching 
    induced during model fitting in the MCMC model.
    \makenote{mention label switching issue earlier?}
    An alternative that works for both the MCMC model and variational model
    is to consider pairwise coincidence of observations in clusters.
    We define a \emph{Coincidence}\makenote{this or similarity?} matrix $C$ as
    denoting the pairwise posterior probability of cluster coincidence--that is,
    probability of being in the same cluster.  With a sample of size $S$ from
    the posterior of the fitted model, we can calculate this as
    \begin{equation}
        \label{eqn:coincidencemat}
        C_{i,j} = \frac{1}{S}\sum_{s = 1}^S\mathbb{1}_{\delta_{i,s} = \delta_{j,s}}
    \end{equation}
    where $i,j$ are iterating over observations. This reduces the size of 
    information in the posterior conditional distribution of $\bm{\delta}$
    to the lower-diagonal of an $n\times n$ matrix.  Perhaps more importantly,
    this creates a network graph between observations, where the edge weight
    denotes the posterior probability of cluster coincidence.

There is a rich field of literature on assigning cluster labels to a netork graph.
    






















% EOF