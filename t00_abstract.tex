In order to account for the model flexibility required by the multivariate
    peaks-over-threshold scenario, a Pitman-Yor process mixture model using
    the projection of independent gamma random variables onto the unit 
    hypersphere under the $\mathcal{L}_p$ norm was developed.  This quickly
    presents issues, as the computational burden for MCMC inference in a BNP
    model scales linearly with sample size, and dimensionality.
    An alternative approach based on mean-field variational inference was explored,
    but it has become apparent that fidelity suffers under the mean-field model
    in a high-dimensional setting.  Further exacerbating the problem is the
    high-dimensional setting itself, which leads to a loss of fidelity even for
    MCMC inference, setting aside the computational burden.  A regression model
    is proposed, that in effect establishes a low-dimensional representation of the 
    high-dimensional output space.  This works to ameliorate the lack of granularity
    imposed under the full-dimensional model, but imposes an additional computational
    burden.  We apply these models to a dataset of storm surge simulations 
    derived from the \emph{Sea, Lake, and Overland Surges due to Hurricanes} 
    (SLOSH) model.
    
% EOF