% In order to account for the model flexibility required by the multivariate
%     peaks-over-threshold scenario, a Pitman-Yor process mixture model using
%     the projection of independent gamma random variables onto the unit 
%     hypersphere under the $\mathcal{L}_p$ norm was developed.  This quickly
%     presents issues, as the computational burden for MCMC inference in a BNP
%     model scales linearly with sample size, and dimensionality.
%     An alternative approach based on mean-field variational inference was explored,
%     but it has become apparent that fidelity suffers under the mean-field model
%     in a high-dimensional setting.  Further exacerbating the problem is the
%     high-dimensional setting itself, which leads to a loss of fidelity even for
%     MCMC inference, setting aside the computational burden.  A regression model
%     is proposed, that in effect establishes a low-dimensional representation of the 
%     high-dimensional output space.  This works to ameliorate the lack of granularity
%     imposed under the full-dimensional model, but imposes an additional computational
%     burden.  We apply these models to a dataset of storm surge simulations 
%     derived from the \emph{Sea, Lake, and Overland Surges due to Hurricanes} 
%     (SLOSH) model.
%     \bruno{\bf This abstract needs a complete rewrite. The focus of the paper is to illustrate
%     how to perform multivariate PoT extreme value analysis for lagre dimensional problems.
%     The motivating illustration is the SLOSH. The abstract should contan a two sentence description 
%     of the PoT model, and then deal with the computational problem.}
    
To account for the model flexibility required by a multivariate peaks-over-threshold
    scenario, a Pitman-Yor process mixture model using the projection of independent 
    gamma random variables onto the unit hypersphere underthe $\mathcal{L}_p$ norm was 
    developed.  In this paper, we explore the application of this model to a
    high-dimensional setting, using storm surge at multiple locations as simulated 
    under the \emph{Sea, Lake, and Overland Surges due to Hurricanes} (SLOSH) model.
    As the computational burden for MCMC inference rises linearly with sample size 
    and dimensionality, we consider a variational approach.  It becomes apparent that
    model fidelity suffers under a mean-field model in a high-dimensional setting.
    Further exacerbating this loss in model fidelity as dimensionality increases, is
    the tendency even under MCMC of the fitted model to degenerate to a single extant 
    cluster.  To ameliorate this loss of granularity, a regression model is proposed, 
    that invokes a low-dimensional representation of the output space.  We use these
    models to explore storm surge at sites of critical infrastructure in the Delaware 
    Bay watershed.


% EOF