\subsection{SLOSH}
\makenote{a lot of this is fluff.}
Storm surge inundation is localized flooding, defined as water height above
    ground level, that arises as a result of a storm pushing sea-water onto
    land. Its effect can be catastrophic.  Setting aside potential 
    for direct loss of life, 
    flooding can impose costly damage to property: inundation of homes and 
    businesses destroys possessions, and damages buildings through saturating 
    walls and eroding foundations.  Corrosion resulting from high--salinity 
    flooding can create more long--term damage.\makenote{expand}\needcite  
    Flooding can damage vehicles, such that a single storm can force insurance 
    companies to declare large quantities of vehicles as total losses. 
    \makenote{expand}\needcite.  Flooding damages agriculture: beyond 
    destruction of currently growing or stored crops, or the drowning of 
    livestock, inundation by storm surge results in the ground absorbing salt, 
    affecting the production capacity of the field until abatement. Flooding can
    damage infrastructure: flooded roads can be washed out or have their 
    foundation damaged, flooded sewers and sewage treatment plants can release 
    their contents above ground imposing additional environmental costs.  
    Flooded power infrastructure, such as transformers can short out causing 
    additional damage. \citep{hutchings2021}.  Inundation can impose additional 
    burdens in the moment:  inundation negatively affects the quality of 
    emergency services, such as a hospital being rendered unable to intake 
    patients.  Sufficient flooding may even render a provider entirely out of 
    commission.

Sea, Lake, and Overland Surges from Hurricanes \citep{jelesnianski1992} is a 
    computer model developed by the National Weather Service to simulate storm 
    surge, and its associated inundation caused by hurricanes.   Given storm 
    characteristics, the model takes into account local topology, bathymetry, 
    and surge management devices such as levees, to generate a spatial field of 
    inundation---the maximum observed height of water above ground level 
    (or above normal water level for a data point in a body of water) over the 
    duration of the storm at a location.   These storm characteristics include 
    data pertaining to the eye of the storm when it made landfall---bearing, 
    velocity, latitude, minimum atmospheric pressure of the storm when it 
    made landfall, and projections of sea level rise over time.  A 
    \emph{simulation} from the model, for the basin we are analyzing, is a grid 
    containing \num{23119800} elements with a spatial resolution of 
    \num{0.001} degrees, or approximately 90 meters, covering an area extending 
    from Virginia Beach, Virginia, to Long Island, New York.  We should note
    that the latitude at which the storms are simulated to make landfall are all
    between \num{38.3} and \num{39.3} degrees, roughly the area around the entrance of
    Delaware Bay.
    We have \num{4000} such simulations, produced by SLOSH from a sample of 
    storm characteristics. 

\begin{figure}[t!]
    \centering
    \begin{subfigure}[t]{0.48\textwidth}
        \centering
        \includegraphics[width=0.99\linewidth]{./plots/slosh1run_loghist}
        \caption{Grid output from one storm simulation in SLOSH, with a marginal
            histogram (with log-scale counts) of surge levels in that storm 
            (truncated at 9 feet).\label{fig:slosh1run}}
    \end{subfigure}%
    ~ 
    \begin{subfigure}[t]{0.48\textwidth}
        \centering
        \includegraphics[width=0.99\linewidth]{./plots/sloshthreshold_loghist}
        \caption{
            90th percentile of storm-surge simulations at \num{5283} selected 
            locations with marginal histogram (with log-scale counts).
            \label{fig:sloshthreshold}}
    \end{subfigure}
    \caption{Exploration of SLOSH simulation data.\label{fig:sloshexplore}}
\end{figure}

This paper analyzes SLOSH simulations under an extreme value theory (EVT) framework, 
    using a peaks-over-threshold model.  EVT is a branch of statistics
    that focuses on the tails of the distribution---low density regimes where,
    in this application, the \emph{worst} outcomes occur.  
    We stress here a caveat. EVT assumes that the originating data are independent
    and identically distributed; SLOSH simulations do not meet the second criterion.
    They arise as a result of partially stochastic simulation given a set of input
    parameters---the storm characteristics.  Those storm characteristics themselves
    are sampled via Latin hypercube to fill the allowable parameter space.  That
    having been said, application of the EVT framework to SLOSH still provides us
    with a great deal of information.  It is with that caveat that we continue analysis.
    Figure~\ref{fig:sloshexplore} provides a visual depiction of the SLOSH 
    simulation data.  Figure~\ref{fig:slosh1run} indicates the output of a single 
    storm surge simulation using the SLOSH model. To the right of the map is a 
    histogram (with log-scale counts) displaying observed levels of inundation 
    in the storm, color-coded by the height of the surge at that location.  In this
    plot, there were 45 locations with surge greater than 9 feet above ground level,
    with the maximum value being approximately 19 feet above.  Such occurances
    are highly localized, and are not visible at this scale, so we truncated surge values
    in this plot at 9 feet.  We should also 
    note here that a storm takes some time to occur, and thus values recorded at 
    two locations in a single storm are not necessarily simultaneous.
    In Figure~\ref{fig:sloshthreshold}, we have selected SLOSH grid 
    cells that are in the vicinity of physical features, or locations, of 
    interest.  Displayed are the 90th percentile 
    of inundation for SLOSH simulations at each of those physical features.
    We make clear now, that this analysis is primarily concerned with the
    inferring and applying the dependence structure between storm surge at these locations,
    for storm simulations
    that are in excess of this threshold in at least one identified location.
    Our goal is then a consistent and performant model for multivariate extremes, 
    such that we can learn the dependence structure of extremes in the inundation field.

\makenote{I need to expound upon the necessity/applicability of extreme analysis
    to inundation.  Mention: importance of tails of the distribution, relevance of
    extremes to such analysis, }

The paper proceeds as follows:  Section~\ref{ref:review} details the background 
    for the relevant modelling methods we will be using in this analysis.  In 
    particular, Section~\ref{ref:evt} 
    provides an overview of extreme value theory, to the justification for 
    separating the magnitude of a multivariate extreme from its angular 
    component; Section~\ref{ref:pg} describes the process of creating an angular 
    distribution as well as introducing the model we use in our analysis; and 
    Section~\ref{ref:varbayes} introduces variational Bayesian methods, which we 
    can use to apply our model at large scale. \makenote{rewrite}
    Section~\ref{ref:results} presents our analysis, first demonstrating the 
    efficacy of variational methods on simulated data as compared to MCMC, then
    presenting some interesting results. Finally, Section~\ref{ref:conclusion} 
    concludes.

% EOF 