%% ABSTRACT %%
% In order to account for the model flexibility required by the multivariate
%     peaks-over-threshold scenario, a Dirichlet process mixture model using
%     the projection of independent gamma random variables onto the unit 
%     hypersphere under the $\mathcal{L}_p$ norm was developed.  This very
%     quickly presents computational issues, as the computational burden for
%     MCMC inference scales superlinearly with sample size, and will scale
%     either linearly or exponentially depending on choice of centering 
%     distribution. We propose an alternate approach, developing a 
%     variational Bayes approach for model inference.  We apply this model
%     to a dataset of simulations of storm surge; comprised of 5 thousand
%     observations at more than 3 million locations.

\subsection{SLOSH}
Sea, Lake, and Overland Surges from Hurricanes \citep{jelesnianski1992} is a 
    computer model developed by the National Weather Service to simulate storm 
    surge, and its associated inundation caused by hurricanes.   Given storm 
    characteristics, the model takes into account local topology, bathymetry, 
    and surge management devices such as levees, to generate a spatial field of 
    inundation---the maximum observed height of water above ground level 
    (or above normal water level for a data point in a body of water) for the 
    duration of the storm at a location.   These storm characteristics include 
    data pertaining to the eye of the storm when it made landfall---bearing, 
    velocity, and latitude, minimum atmospheric pressure of the storm when it 
    made landfall, and projections of sea level rise over time.  A 
    \emph{simulation} from the model, for the basin we are analyzing, is a grid 
    containing \num{23119800} 
    elements wih a spatial resolution of \num{0.001} degrees, or approximately 
    90 meters, \makenote{(verify)}, 
    covering an area extending from Virginia Beach, Virginia, to Long Island, 
    New York.  We have \num{4000} such simulations, produced by SLOSH from a 
    sample of storm characteristics.

\makenote{Insert graphic of coverage of model}

Inundation can be catastrophic.  Setting aside the potential for loss of life, 
    flooding can impose costly damage to property: inundation of homes and 
    businesses destroys possessions, damages buildings through saturating 
    walls and eroding foundations.  Corrosion resulting from high--salinity
    flooding can create more long--term damage.\makenote{expand}\needcite  
    Flooding can damage vehicles, such that a single storm can force insurance 
    companies to declare large quantities of vehicles as total losses. 
    \makenote{expand}\needcite.  Flooding damages agriculture: beyond 
    destruction of currently growing or stored crops, or the drowning of 
    livestock, inundation by storm surge results in the ground absorbing salt, 
    affecting the production capacity of the field until abatement. Flooding can
    damage infrastructure: flooded roads can be washed out or have their 
    foundation damaged, flooded sewers and sewage treatment plants can release 
    their contents above ground imposing additional environmental costs.  
    Flooded power infrastructure, such as transformers can short out causing 
    additional damage. \citep{hutchings2021}.  Inundation can impose additional 
    burdens in the moment:  inundation negatively affects the quality of 
    emergency services, such as a hospital being rendered unable to intake 
    patients.  Sufficient flooding may even render a provider entirely out of 
    commission.

Unfortunately, with respect to flooding in general, SLOSH provides only an 
    incomplete picture.  It does not account for rainfall or rivers bursting
    their banks.

\makenote{I need to expound upon the necessity/applicability of extreme analysis
    to inundation.  Mention: importance of tails of the distribution, relevance of
    extremes to such analysis, }

Our goal is then a consistent and performant model for multivariate extremes, 
    such that we can learn the dependence structure of the inundation field.


Section~\ref{ref:review} details the background for the relevant modelling methods
    we will be using in this analysis.  In particular, \ref{ref:evt} provides
    an overview of extreme value theory, to the justification for separating
    the magnitude of a multivariate extreme from its angular component; \ref{ref:pg}
    describes the process of creating an angular distribution as well as introducing
    the component distribution we will be using in our Bayesian non-parametric 
    mixture model, which is further expanded in \ref{ref:npbayes}; and 
    \ref{ref:varbayes} introduces variational Bayesian methods, which we can use
    to apply our model at large scale.  Section~\ref{ref:methodology} 
    \makenote{does something.  probably most of review should be moved here?  
    but none of that is new, only new thing is the particular variational model.}
    Section~\ref{ref:results} presents the results of our analysis, and finally
    Section~\ref{ref:conclusion} concludes.

% EOF 