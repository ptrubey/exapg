\subsection{Posterior clustering of storms}
One potential application leveraging the use of a Bayesian non-parametric prior is 
    exploiting the clustering of observations with similar characteristics.  
    Recall that $\delta_i$ is the cluster identifier for observation $i$. Within the 
    MCMC considered in this work $\delta_i$ is explicitly sampled, though \st{it is integrated out}
    in our variational \st{model} \bruno{approach}. \bruno{Notice that, given
    samples of the posterior distribution of }\st{In posterior analysis, given} 
    $\bm{\alpha}$, and $\bm{\pi}$, \st{where} \bruno{define} 
    $\pi_j = \nu_j\prod_{k = 1}^{j-1}(1 - \nu_k)$. \bruno{Then,} the conditional 
    probability of cluster  assignment can be calculated as
    \begin{equation}
        \label{eqn:clusterprob}
        \text{P}\left(\delta_i = j\mid\bm{\alpha},\bm{\pi},\bm{y}_i\right) 
            = \frac{\pi_j\mathcal{PG}(\bm{y}_i\mid\bm{\alpha}_j,\bm{1})}{
            \sum_{k = 1}^J \pi_j\mathcal{PG}(\bm{y}_i\mid\bm{\alpha}_k,\bm{1})}.
    \end{equation}
    This approach relies on samples of $\bm{\alpha},\bm{\nu}$ from a model fitted without 
    cluster labelling. This can present a problem, as interpretation of clusters requires 
    labelling, or a fixed assignment of observations to clusters, to avoid label switching.
    A label-switch between clusters $a$ and $b$ occurs when the parameters of clusters 
    $a$ and $b$ swap, causing the bulk of observations formerly in cluster $a$ to be 
    classified in cluster $b$, and vice-versa. This issue is not present in a variational 
    implementation, as, once the variational distribution is obtained, the distributions 
    of cluster parameters are fixed.  In practice, we find that the label switching 
    concern may be overstated.  For our example, in the MCMC approach,
    after the sampler reaches convergence, we find the posterior distribution of 
    cluster assignments and resulting cluster parameters to be relatively stable.

With that being said, a means of estimating the cluster label of a given observation that 
    is consistent between MCMC and variational approaches is to take samples of $\bm{\pi}$, 
    and $\bm{\alpha}$, and then sample $\bm{\delta}_i$ following Equation~\eqref{eqn:clusterprob}.  
    Using this approach, we group observations by their sampled cluster identifier, and count 
    the number of emergent clusters in the data.

% EOF