\subsection{Posterior clustering of storms}
One potential application leveraging use of a Bayesian non-parametric prior is 
    exploiting the clustering inherent to the method to identify emergent clusters of
    data---observations that act similarly.  Recall $\delta_i$ is the 
    cluster identifier for observation $i$.  Sampling of $\delta_i$ is made explicit in
    the MCMC model, though it is integrated out in our variational model. In
    posterior analysis, given $\bm{\alpha}$, and $\bm{\pi}$, where 
    $\pi_j = \nu_j\prod_{k = 1}^{j-1}(1 - \nu_k)$, a conditional probability of cluster 
    assignment can be calculated as
    \begin{equation}
        \label{eqn:clusterprob}
        \text{P}\left(\delta_i = j\mid\bm{\alpha},\bm{\pi},\bm{y}_i\right) 
            = \frac{\pi_j\mathcal{PG}(\bm{y}_i\mid\bm{\alpha}_j,\bm{1})}{
            \sum_{k = 1}^J \pi_j\mathcal{PG}(\bm{y}_i\mid\bm{\alpha}_k,\bm{1})}.
    \end{equation}
    This approach relies on samples of $\bm{\alpha},\bm{\nu}$ from the fitted
    model to generate \emph{loose} clusters for which there is no inherent 
    labelling.  This can present a problem, as interpretation of clusters first 
    requires \emph{labelling}, or a fixed assignment of observations to clusters.

The variational implementation would avoid a label-switching issue that may be inherent 
    to MCMC sampling.  A label-switch between clusters $a$ and $b$ has occurred when
    the parameters of clusters $a$ and $b$ have nearly swapped, such that the
    bulk of observations formerly in cluster $a$ are now in cluster $b$, and 
    visa-versa.  This issue does not completely follow to a variational 
    implementation, as post-fitting, the distributions of cluster parameters are 
    fixed.  In practice, we find this concern may be overstated.  In the MCMC model,
    after the sampler reaches convergence, we find the posterior distribution of 
    cluster assignments and resulting cluster parameters to be relatively stable.

With that being said, a means of ascertaining the emergent cluster label of a
    given observation that is consistent between MCMC and variational approaches
    might be to take samples of $\bm{\pi}$, $\bm{\alpha}$, and then sample
    $\bm{\delta}_i$ as per Equation~\eqref{eqn:clusterprob}.  Via this method, we
    we group observations by their sampled cluster identifier, and count the number
    of emergent clusters in the data.

% EOF